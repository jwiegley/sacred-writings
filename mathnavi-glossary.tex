% Mathnavi Glossary - Persian/Arabic Terms with English Translations
% Based on hidden-words-glossary.tex format
% Using IJMES transliteration standard
% For use with mathnavi.tex bilingual edition

% Macro definitions (should match hidden-words-glossary.tex)
\newcommand{\define}[3]{\textfarsi[voc]{\Large
    \textbf{#1}}\hspace{3mm}{\headwordfont \large
    \textit{#2}}\hspace{3mm}{\Large #3} \\[3ex]}
\newcommand{\fulldefine}[3]{\textfarsi[voc]{\Large
    \textbf{#1}}\hspace{3mm}{\headwordfont \large
    \textit{#2}}\vspace{-1ex}\begin{quote}\Large #3\end{quote}\vspace{1ex}}

% ============================================================================
% DIVINE ATTRIBUTES & NAMES
% ============================================================================

\fulldefine{حَیاتُ العَرش}{ḥayāt al-'arsh}{Life of the Throne. Theological term referring to the divine presence sustaining all existence. Used as a title of veneration.}

\define{خُورشیدِ وِداد}{khurshíd-i-vidád}{brilliant sun of love}

\define{فَخرِ زَمان}{fakhr-i-zamán}{glory of the age}

\define{مَلیکِ عَرش}{malík-i-'arsh}{Sovereign of the throne}

\define{میرِ دِیار}{mír-i-diyár}{Leader of the lands}

\define{شاهِ مِهتَر}{sháh-i-mihtar}{great King}

\fulldefine{بَهاءُ الله}{Bahá'u'lláh}{The Glory of God. Proper name of the Prophet-Founder of the Bahá'í Faith (1817-1892). Author of this Mathnavi.}

\define{اللّه}{Alláh}{God (the one God)}

% ============================================================================
% MYSTICAL & THEOLOGICAL CONCEPTS
% ============================================================================

\fulldefine{لِقا}{liqá'}{Meeting, encounter with the divine. In mystical context refers to attaining the presence of God or recognizing the Manifestation.}

\fulldefine{سِرِّ بَقا}{sirr-i-baqá}{Secret of eternity, mysteries of eternal reunion. The eternal divine mysteries that transcend temporal existence.}

\fulldefine{کَوثَر}{kawthar}{Kawthar. The river or fountain of Paradise mentioned in the Qur'án (108). Symbolizes divine bounty and spiritual abundance.}

\define{اَنوار}{anvār}{lights, illuminations}

\fulldefine{اَنوارِ طور}{anvār-i-ṭúr}{Lights of Mount Sinai. Reference to the theophany experienced by Moses on Mount Sinai, symbolizing divine revelation.}

\fulldefine{تیغِ اللّهیَت}{tígh-i-ilāhíyat}{Sword of Godhead. Divine power and justice that vanquishes falsehood and establishes truth.}

\fulldefine{نارِ رَبّانیَت}{nār-i-rabbāníyat}{Divine fire. Transformative spiritual energy that purifies and elevates the soul. Fire symbolizing divine love.}

\fulldefine{سِدرَه}{sidra / sidrat}{The Lote Tree, specifically Sidrat al-Muntahá (Lote Tree of the utmost boundary). Qur'ánic reference (53:14-16) to the celestial tree marking the boundary beyond which none may pass. Symbol of the Manifestation of God.}

\fulldefine{عَرش}{arsh / 'arsh}{The Throne. Divine Throne symbolizing God's sovereignty and majesty. In mystical interpretation, represents the heart of the believer or the station of the Manifestation.}

\fulldefine{طور}{ṭúr}{Mount Sinai (Arabic: Ṭúr Síná). The mountain where Moses received the divine revelation and beheld the Burning Bush.}

% ============================================================================
% CORE SPIRITUAL VOCABULARY (High Frequency)
% ============================================================================

\define{جان}{ján}{soul, life, spirit}

\define{عِشق}{'ishq}{divine love, mystical love}

\define{نور}{nūr}{light, divine illumination}

\define{روح}{rúḥ}{spirit, soul, breath}

\fulldefine{مَعنى}{ma'ná}{Meaning, inner meaning, spiritual reality. The esoteric or mystical significance underlying outward forms and words.}

\define{اَسرار}{asrár}{mysteries, secrets}

\define{نار}{nār}{fire (often divine fire)}

\define{بَقا}{baqá}{eternity, permanence, everlasting}

\define{وَصف}{vaṣf}{description, attributes}

% ============================================================================
% SPIRITUAL STATES & STATIONS
% ============================================================================

\fulldefine{مَجنون}{majnún}{Crazed, mad with love. In Sufi poetry, refers to one enraptured by divine love, following the archetype of Majnún (lover of Laylá). Translated as ''enthralled'' or ''enamored'' in archaic style.}

\define{مَست}{mast}{intoxicated, drunk (spiritually)}

\define{مَرهون}{marhún}{pledged, indebted, beholden}

\define{بیهوش}{bí-húsh}{unconscious, senseless, enraptured}

\define{فانى}{fání}{annihilated, effaced, dissolved}

\define{باقى}{báqí}{remaining, abiding, everlasting}

\define{پاک}{pák}{pure, sacred, holy}

\define{عَیان}{'iyán}{manifest, evident, visible}

% ============================================================================
% VEILS & BARRIERS
% ============================================================================

\fulldefine{پَرده}{parda}{Veil, curtain. That which conceals spiritual reality from perception. Removing the veils is a central theme in mystical literature.}

\fulldefine{حِجاب}{ḥijáb}{Veil, barrier. That which separates the seeker from the divine Beloved. Can refer to attachment to worldly things or the limitations of human understanding.}

% ============================================================================
% COSMIC & EXISTENTIAL TERMS
% ============================================================================

\define{جَهان}{jahán}{world, universe, cosmos}

\define{زَمان}{zamán}{time, age, epoch}

\define{میان}{miyán}{midst, between, among}

\define{هَستى}{hastí}{existence, being}

% ============================================================================
% VISION & PERCEPTION
% ============================================================================

\define{چَشم}{chashm}{eye, vision, sight}

\define{روى}{rúy}{face, countenance}

\fulldefine{بَیان}{bayán}{Utterance, exposition, declaration. Can refer to the revealed Word or to the act of making spiritual truths manifest through speech.}

% ============================================================================
% POETIC METAPHORS (Soul's Journey)
% ============================================================================

\define{ذَرّه}{dharra}{atom, mote (metaphor: seeker)}

\define{قَطره}{qaṭra}{drop (metaphor: individual soul)}

\define{بَحر}{baḥr}{ocean, sea (metaphor: divine vastness)}

\define{دانه}{dāna}{seed, grain (metaphor: potential)}

\define{سَما}{samá}{sky, heaven (metaphor: divine realm)}

\define{کوه}{kúh}{mountain (metaphor: spiritual height)}

\define{چاه}{cháh}{well, pit, abyss}

% ============================================================================
% MANIFESTATION & ACTION
% ============================================================================

\define{پَدید}{padíd}{manifest, appear, become visible}

\fulldefine{عَصا}{'aṣá}{Staff, rod. Particularly refers to the staff of Moses which became a serpent and performed miracles (Qur'án 7:107, 20:20).}

% ============================================================================
% PROPHETIC NAMES & FIGURES
% ============================================================================

\fulldefine{یوسُف}{Yúsuf}{Joseph, son of Jacob. Prophet renowned for his beauty and his trials (imprisonment, false accusation). Qur'ánic chapter 12 narrates his story. Symbol of divine beauty and patient endurance.}

\fulldefine{موسى}{Músá}{Moses. Prophet who spoke with God on Mount Sinai, received the Torah, and led his people from bondage. Known as Kalímu'lláh (Interlocutor of God).}

\fulldefine{کَلیم}{Kalím}{Interlocutor, one who speaks with God. Title specifically referring to Moses (Kalímu'lláh) who conversed with God on Mount Sinai.}

\fulldefine{حُسَین}{Ḥusayn}{Ḥusayn, grandson of the Prophet Muhammad, martyred at Karbala (680 CE). Supreme symbol of martyrdom and sacrifice in Islamic tradition.}

\fulldefine{حَبیب}{Ḥabíb}{Beloved. Title often applied to the Prophet Muhammad (Ḥabíbu'lláh, Beloved of God).}

\fulldefine{روحُ الله}{Rúḥu'lláh}{Spirit of God. Title of Jesus Christ in Islamic tradition, emphasizing His spiritual nature and divine station.}

% ============================================================================
% ADDITIONAL SIGNIFICANT TERMS
% ============================================================================

\define{یار}{yár}{Friend, Beloved (the divine Friend)}

\define{دِل}{dil}{heart}

\define{شاه}{sháh}{king, sovereign}

\define{امر}{amr}{Cause, Command, divine revelation}

\define{لُطف}{luṭf}{grace, kindness, divine favor}

\define{فَضل}{faḍl}{bounty, grace, favor}

\define{جَمال}{jamál}{beauty (divine beauty)}

\define{کَربَلا}{Karbalá}{Karbala, site of Ḥusayn's martyrdom}

\fulldefine{قِبطیان}{qibṭíyán}{Egyptians, Copts. In reference to Moses, those who opposed him. Symbolizes opponents of divine truth.}

% ============================================================================
% USAGE NOTES
% ============================================================================
%
% 1. Transliteration follows IJMES (International Journal of Middle Eastern
%    Studies) standard as used in hidden-words-glossary.tex
%
% 2. Use \define{}{}{} for simple term definitions
%    Use \fulldefine{}{}{} for terms requiring extended explanation
%
% 3. Capitalization of divine pronouns in translations:
%    - Capitalize when referring to Manifestation: Thou, Thee, Thy, Thine
%    - Capitalize God and related proper nouns
%
% 4. Archaic English style preferred for devotional tone:
%    - thou/thee (not you)
%    - thy/thine (not your)
%    - hadst/wert (not had/were)
%    - Till (not So that)
%
% 5. Translation philosophy:
%    - Interpretive poetic rendering preferred over literal
%    - Preserve metaphorical and mystical meanings
%    - Maintain elevated devotional tone
%
% 6. For integration with mathnavi.tex:
%    - Ensure XeLaTeX compilation
%    - Use arabxetex package for Persian text
%    - Reference with \textfarsi[voc]{} for vocalized text
%
% ============================================================================
% REFERENCES
% ============================================================================
%
% Source texts:
% - mathnavi-persian.txt (955 lines, 318 couplets)
% - mathnavi-start.txt (lines 1-43, translation exemplar)
%
% Style references:
% - hidden-words-glossary.tex (IJMES standard, macro format)
% - seven-valleys.tex (archaic English precedents)
%
% Documentation:
% - .taskmaster/reports/mathnavi-baseline-report.md
% - .taskmaster/docs/mathnavi-terminology-guide.md
%
% Compiled: 2025-11-24
% Project: Mathnavi Translation (Bahá'u'lláh's revealed Writings)
% Translator: Following Shoghi Effendi style conventions
%
% ============================================================================
