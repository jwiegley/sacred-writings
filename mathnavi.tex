مثنوى مبارك

آثار قلم اعلى، ۱۵۹ بديع، جلد ۲، صفحه ۳۰۸ - ۳۲۰

﴿ هوالأبهى ﴾
He is the All-Glorious!

اى حيات العرش خورشيد وداد
O life of the Throne, O brilliant sun of love
که جهان و امکان چه تو نورى نزاد
Whose light outshines the world and all its powers

گر نبودى خلق محجوب از لقا
Hadst thou not been, creation wert denied His Face
يک دو حرفى گفتم از سرّ بقا
Thus a word or two on the mysteries of eternal reunion

تا که جانها جمله مرهونت شوند
Till every soul is pledged to thee
تا که دلها جمله مجنونت شوند
And every heart enthralled by thee

تا ببينى عالمى مجنون و مست
Till all the world is drunk and crazed with love
روحها به هر نثار اندر دو دست
Their souls in hand, ready to be cast aside

تا رسد امر تو اى فخر زمان
Till thy cause appear, O glory of the age
بر فشانند بر قدومت رايگان
And all scatter at Thy sudden arrival

سر بر آر از کوه جان خورشيد وار
Till thou rise above the mountain, soul bright-shining as the sun
تا ببينندت عيان از هر کنار
And they behold thee more clearly than all things else

جلوه ده آن روى همچون ماه را
Thy countenance streaming forth like the moon
سبز و خُرَّم کن ز لطف اين گاه را
Making this place verdant and fresh by thine approach

قطره مى جويد ز بحرت کوثرى
The drop seeks from thine ocean a river
کوثرى کن ز انکه شاه مهترى
Let flow the river, for thou art the great King

ذرّه گشته ملتمس نور تو را
وا دهش از لطف بيچون و چرا

دانه بگشاده دهان سوى سما
تا بيايد بر وى از فضلت بها

قطره هاى رحمتت بر وى ببار
اى مليک عرش و اى مير ديار

خرق کن اين پردهء صد توى را
خوش تماشا ده کنون آنروى را

زانکه در فضلت نباشد شبهه اى
بهر ما بر بند ز لطفت توشه اى

مشرق کل کن کنون اين غرب را
بهجت مل ده کنون اين شرب را

نور دل را نور ده ز انوار نور
تا ببينند از رخت انوار طور

هان بکش آن تيغ اللّهيت را
هين بکش اين دشمنان دينت را

بر فروزان نار ربّانيت را
خوش بسوزان ملحد حربيت را

جمله خفّاشند اى خورشيد روز
سر بر آر و جمله ظلمانى بسوز

صاف کن اين درد غم آلوده را
نور ده اين شمع شب افسرده را

عالمى قائم بتو چون تو بجان
تا شود پيدا ز امرت کن فکان

اى بهاى جان بياد روى تو
نکته هاگويم همى از خوى تو

تا بر آرم جانها را از خرد
تا ببينم درّ عشقت که خرد

بر فروزم آتشى اندر جهان
تا بسوزم پرده هاى قدسيان

حور معنى را بر آرم از حجاب
نور غيبى را کنم کشف نقاب

رمزى از اسرار عشق سرمدى
باز گويم چون بجان باز آمدى

خوش بيا اى طير نارى در بيان
تا نماند وصف هستى در ميان

پاک کن اين قلبهاى پر حسد
نقد کن اين قلب هاى بى رصد

تا که بيهوشان عهدت اى کريم
هم بهوش آيند از جام قديم

بلکه از الحان قدس اى يار ما
دورکن هم هوش و بى هوشى زما

اى سرافيل بها اى شاه جان
يک حياتى عرضه کن بر مردگان

سدره اوّل بود ز اغصان دل
وا رهانش از هوا و آب و گل

تا ز جوهر و ز عرض فارغ شود
تا ز شمعش شمسها بازغ شود

اين نهالت غرس کن در ارض دل
پس مقدّس دارش از اشراق و ظل

هم تو حفظ از مختلف بادش نما
هم ز وهم مشرک آزادش نما

اصل او ثابت نما در ارض جان
فرع او را بگذران از آسمان

نو بهارى تو ز نو آور عيان
تا زحشرت بر جهند اين مردگان

جوش درياهاى عشق از جوش تو
هوش اطيار بقا از هوش تو

بوى پيراهن بوز از مصر جان
سدره موسى نما اينجا عيان

اى نگار از روى توآمد بهار
زين بهار آمد حقايق بيشمار

هرگل از وى دفترى ازحسن دوست
هر دل از وى کوثرى از فضل هوست

اين بهاران را خزان نايد ز پى
جمله گلها طائف اندر حول وى

اين بهارى نه که جان‌ درکش کند
اين بهارى که روانها را کند

آن بهاران شوق خوبان آورد
و اين بهاران عشق يزدان آورد

آن بهاران را فنا باشد عقب
و اين بهاران را بقا باشد لقب

آن بهار از فصل خيزد در جهان
و اين بهار از نور روى دلستان

آن بهاران لاله ها آرد برون
و اين بهاران ناله ها دارد کنون

اين بهار سرمدى از نور شاه
بر زده خرگاه تا عرش اله

جمله در خرگاه او داخل شدند
گر تو چشمت هست بنگر هوشمند

شاه ما چون پرده از رخ بفکند
اين بهاران خيمه بر گردون زند

يار ما چون بفکند از رخ نقاب
اين بهاران بر فروزد بى حجاب

ما برويش در بهاران اندريم
ما ز رويش در گلستان ننگريم

ما بذکرش فارغيم از ذکر کان
ما ز شمسش بازغيم اندر جهان

گر نسيمى بر وزد زين خوش بهار
يوسفان بينى که آيند در نظار

گر نسيمى بر وزد زين بوستان
يوسفان روح بينى در جهان

جسمها بينى که گردد همچو روح
روح را هر دم رسد صد گون فتوح

اين ربيع قدس جانان هر دمى
صد بيان دارد ولی کو محرمى

اين بيان باشد مقدّس از لسان
کى بمعنيش رسند اين ناکسان

اين بيان از گفت و لفظ و صوت نيست
اين بيان جانست و اورا موت نيست

عاشقان بينى تو اندر اين بهار
جان نثار آورده هر دم صد هزار

اين بهار عزّ روحانى بود
اين ربيع قدس ربّانى بود

گر وزد بر تو نسيمى زين سبا
جان فانيّت کشد جام بقا

گرنسيمى آيدت از کوى دوست
جان فدايش کن که اينجان هم از اوست

لالهء توحيد بين در اين بهار
سنبل تجريد بين از زلف يار

غنچه هاى معرفت زين طرف جو
جملگى از شوق او در جستجو

سروهايش حاکى از قدّ نگار
سبزه هايش دفترى از خدّ يار

بلبلانش مست از جام الست
قمريانش از جمال دوست مست

عندليبان در هواى وصل او
جمله مستند از نسيم فضل هو

نغمه اين بلبل ار ظاهر شود
جان خلقان از حسد طاهر شود

بحر معنى زين بيان موّاج شد
فلک هستى زين کرم لجلاج شد

هر شقائق که بر آيد زين بهار
صد حقايق بر دمد از سرّ يار

بوى مشک آيد همى از جعد يار
دست فضلش ميکند بر تو نثار

زلف او همچون سمندر بين بنار
کو همى گردد بنار روى يار

عندليب قدسى از هجران دوست
ناله ها دارد که سوزد مغز و پوست

گر زدرد هجر خود آهى کشد
شعله اندر جان خاصان افکند

غير خاصانرا نباشد زين نصيب
وا مگير از لطف اين فضل اى حبيب

بر وزان مشک الهى را زجان
تا ز عطرت بو برند اين ناکسان

اين بهار روح باشد جاودان
نى بهارى کز پيش آيد خزان

زين بهار قدس روح آيد برون
و ز هوايش نور نوح آيد برون

بر نشاند اهل کشتى را بفلک
پس ببخشد هرکه را صد گونه ملک

اى جمال اللّه برون آ از نقاب
تا برون آيد ز مغرب آفتاب

نافهء علم لدنّى بر گشا
مخزن اسرار غيبى بر گشا

تا ز مشکت بو برند اين مردگان
تا ز خمرت خوش شوند اين بيهشان

اين ذلی لارض وحدت را ز جود
خلعت عزّت بپوشان اى ودود

فانيى را پوش از ثوب بقا
فقر بحتى را چشان شهد غنا

تا برون آيد بکلّى از حجاب
بر درد امکان و هستى را نقاب

بى خود و سرمست آيد او برون
شمع سان اندر زجاج راجعون

چونکه اين خار از گلستانت دميد
صد گلستان آر از وى تو پديد

هر گلستان را باسمى زن رقم
پس بهر برگى نما سر‌ّ قدم

تا که انوار رخت آيد عيان
پر کند نورت زمين و آسمان

بر وزان بادى ز رحمت اى کريم
بر دران احجاب غفلت زين سقيم

در پناه سدرهء خود جاى ده
روحهاى پاک اى سلطان مه

بابى از رضوان معنى بر گشا
سدّ مکن اين باب از بهر خدا

تا درآيم بى حجاب اندر جهان
تا کنم رمزى ز احسانت بيان

گفت اللّه اللّه اى مرد نکو
رمز حق در نزد نادانان مگو

اللّه اللّه اى لسان اللّه راز
نرم نرمک گوى و با مردم بساز

هم مگر لطف توگيرد دستشان
پس کند فارغ ز بيم اين و آن

پر معنى بر گشا طيّار شو
در هواى قرب او سيار شو

قرب او با جان نه در طى قدم
چون بجان پوئى در آئى در قدم

پس به آنى طى افلاک وجود
نيست مشکل چون شوى زاهل سجود

در بيان اين بگويم نکته اى
تا برى از آب حيوان حصّه اى

تا شوى واقف ز رضوان بقاء
تا برى راهى باقليم لقا

تا بطىّ الارض معنى پى برى
تا چه روح اندر هوايش بر پرى

چون تو هستى اين زمان در دام گل
کى برى بوئى تو از رضوان دل

پس برهنه شو تو از ثوب قيود
پس مقدّس کن تو جانرا از حدود

ظلمت دل را ز نورش کن منير
تا شوى در ملک جانها تو امير

چونکه ظلمت رفت نورش مشرق است
بر دلت انوار طورش بارق است

چونکه ليلت رفت صبح آمد پديد
هم نسيم عزّ روحانى وزيد

پس تو اين ظلمات و اين نفس تباه
آب حيوانش تجلّى اله

گر تو زين ظلمات نفست بگذرى
بى تعب از خمر حيوان بر خورى

پس تو اندر ظل خضر جان در آ
تا شوى فارغ از اين ظلمت سرا

آن خضر نوشيد و برهيد از ممات
وين خضر بخشد دو صد عين حيات

آب حيوان بر همه انفاق کرد
خود نموده جان نثار شاه فرد

آن خضر جهدى نمود آنگه رسيد
زين خضر صد چشمه آنى شد پديد

آن خضر شد از پى چشمه دوان
وين خضر را چشمه ها از پى روان

اى بهاى جان تو باز آ زين شکار
تاکنى صيد معانى صد هزار

صيد گورانرا بهل از بهر گور
صيد معنى آر از صحراى طور

صيد کردى جان عشاقان بدشت
تا که جانها جمله از هستى گذشت

نيست فرصت تا تو از اسرار گل
پيش بلبل گوئى اى سلطان گل

بر پران بازى ز ساعد اى نگار
تا که باز آرد معانى زان ديار

اين زمان سيمرغ معنى صيد کن
بر گشا گنجى تو از مفتاح کن

آنچه کردى وعده اکنون کن وفا
اى ز نورت روشن اين ارض و سما

از بهار خود بکن خرّم جهان
تا که رضوانت شود رشک جنان

از حقائق بس شقائق بر دمان
در فضاى اين بهارستان جان

پس ز هر گل رمز بلبل کن عيان
شرح مل در دل بگو با خسروان

زانکه اينجا اين زمان نامحرم است
محرم و نامحرم‌ اينجا چون هم است

اى صباى صبح از زلفين يار
نافه هاى مشک روحانى بيار

اى سحاب فضل روحانى ببار
تا صدف لوءلوء همى آرد ببار

شرح اسرار لَدنّى باز ماند
ذکر طىّ الارض معنى باز ماند

پس تو اى مخمور از جام غرور
نار نفست را بدل ميکن بنور

تا کنى طى جهان در يک نفس
تا رها گردى ز حبس اين قفس

پيش از آن که اندرائى ظل دوست
نى خبر از مغز دارى نى ز پوست

پاى معنيّت بگل باشد فرو
بى خبر ز انوار آن روى نکو

چون بظلّ شاه جان مسکن کنى
آن زمان دل از جهانى بر کنى

اول ساعت بدى اندر تراب
آخر ساعت گذشتى ز آفتاب

پس بآنى طى عالم هاى جان
بى قدم کردى تو اى سالک بدان

اين زمان بوئى ز عطرستان جان
بر وزيد و شد معطر اين جهان

باز مشک جان از آن رضوان جود
بر وزيد و برد جمله آنچه بود

هوش و بى هوشى زدست اينجا برفت
مست و هشيارى همه يکجا برفت

صحو شد هم محو و محوى هم نماند
مست شد هشيار و صحوى هم نماند

آنچه بود از اسم و رسم اين جهان
فانى آمد چونکه شد شاهم عيان

زانکه اسما گر دو صد قرن او پرد
مى نيارد که ز قدرش بو برد

آنچه چشمت ديد و هم گوشت شنيد
او ز جمله پاک آمد اى رشيد

پس تو با اين گوش وچشم اى بى بصر
کى شوى از سرّ جانان با خبر

چشم ديگر بر گشا از يار نو
گوش ديگر باز کن آنگه شنو

چشم جاهل مى نبيند جز قدم
چشم عارف بيند اسرار قِدم

چشم عارف صد هزاران ساله راه
چشم جاهل مى نبيند روى شاه

سائلی مر عارفى را گفت کى
تو بر اسرار الهى برده پى

وى تو از خمر عنايت گشته مست
هيچ يادت آيد از روز الست

گفت ياد آيد مرا آن صوت وگفت
کو بدى بود و نباشد اين شگفت

هست در گوشم همى آواى او
آن صداى خوب جان افزاى او

عارف ديگر که بر تر رفته بود
درّ اسرار الهى سفته بود

گفت آن روز خدا آخر نشد
ما در آن يوميم و آن قاصر نشد

يوم اوباقى ندارد شب عقب
ما در آن روز و نباشد اين عجب

گر رود ذوقش ز جان روزگار
مى نبينى عرش و فرشى بر قرار

زانکه يوم سرمدى از قدرتش
لا يزول امد پديد از حضرتش

پس تو اى جان اين معما گوش دار
پند اسرار الهى هوشدار

تاکه رزق جان برى ازحکمتش
تا که جان سازى فداى طلعتش

تاکه هر دم بشنوى الحان او
تا بنوشى جامى از احسان او

تا شوى واقف تو بر اسرار عشق
تا چشى راح ازل ز انهار عشق

رخ نگردانم ز سيف اين خسان
گر دو صد بارم کشند اين کافران

خمر تو نوشيد جانم ز ابتدا
هم بيادت جان دهم در انتها

اى بها يک آتشى از نو فروز
عالم تحقيق و دانش را بسوز

پاک کن جان را ا ز اوصاف جهان
بر گشا رمزى ز اسرار نهان

موجى از درياى ژرف معنوى
بر فکن تا فلک لفظى بشکنى

يک قدح در ده که تا از خود رهم
همچو صفدر پرده ها را بر درم

اى زاسمت سدره هستى ببار
هم ز دستت قدرت حق آشکار

اى جهانى در کف تقدير تو
منقلب گه ساکن از تدبير تو

نور ده اين شمع و هم زو نور ده
اين جهات مختلف اى شاه مه

اين چراغى را که روشن کرده‍ئى
در زجاج حفظ حفظش کردهئى

هم ز دُهن جود داديّش مدد
و ز فتيله امر کرديّش رشد

پس ز باد ظلم حفظش دار تو
تا شود ظاهر از او انوار تو

دست دشمن از سرش کوته نما
اى تو ماه امر و شاه انّما

بنگر اين شمعت که گشته مبتلا
در ميان گِردباد پر بلا

چون ز انوار جمالت نور يافت
پس مکن در نزد امکانش تو مات

چونکه کردى روشنش خامش مکن
چونکه هوشش دادهئى بيهش مکن

اى ز مهرت ذرّه خورشيدى شود
وى ز قهرت شير عصفورى بود

بر وزيده بادها از هر کنار
مانده اين شمعت ميان اى کردگار

گر تو خواهى آب آتش ميشود
ور نخواهى آتش آندم بفسرد

اى ز حکمت ديو گردد همچه حور
وى ز امرت بر دمد از نار نور

گر تو خواهى باد چون دهنى شود
بر فزايد روح و هم نورى بود

اى بهاء اللّه چه نارت بر فروخت
خرمن هستى عشاقان بسوخت

يک شرر از نار بر دلها زدى
صد هزاران سدره بر سينا زدى

پس ز هر دل سدره ها آمد پديد
مو سيا اينجا بسر بايد دويد

تا که نار اللّه معنى را ز جان
بنگريد و وارهيد از قبطيان

اى ذبيح اللّه ز قربانگاه عشق
بر مگرد و جان بده در راه عشق

بى سر و بيجان بيا در کوى يار
تا شوى مقبول اهل اين ديار

وادى عشق است روح اللّه بيا
با صليب از راه و هم بيره بيا

از فلک بگذر هم از معراج جسم
اى تو شاه جان و هم بهاج جسم

بلبل روحى تو بر گلزار روح
باز ميآئى تو مهماندار روح

ساعد شه مسکنت اى باز جان
سوى مقصد آى اينجا رايگان

پس تو هم اى نوح فلک تن شکن
خويش را در بحر نورانى فکن

غرق کن اين نفس و حفظ خود مخواه
تا برون آرى سر از جيب اله

حفظ خواه از شاه و از کشتى مخواه
تا در آئى در پناه حفظ شاه

هم تو اى موسى بطور جان بيا
بگذر از نعل و ردا عريان بيا

تا شوى واقف تو از اسرار نار
ز انکه نار آمد همى از زلف يار

زلف او نارى که سوزد جان عشق
کفر و ايمان هم سرو سامان عشق

زلف او نارى که بر فاران چمد
هم تبارش گردن دوران خمد

بس کن اى ورقا تو از اسرار نار
لؤلؤ جان پيش اين کوران ميار

اين عصا سيفى بود کز دست حق
مى بدرد صف امکان چون ورق

آن عصا از دوحهء بستان دميد
و اين عصا از امر حق آمد پديد

آن عصا از آب و گل آمد برون
اين عصا از نار دل باشد کنون

اين عصا نارى بود کز شعله اش
مى بسوزد پرده هاى غلّ و غش

اين عصا بادى بود کز قوم هود
ميشناسد موءمن از کافر جحود

کشتى آمد آن عصا در عهد نوح
هم عصا در عهد عيسى گشت روح

موسيا نارت ز جان شعله کشيد
پس بطور جان همى بايد رسيد

نعل چه از جان و از ايمان گذر
همچو باد از ملک‍جان پرّان گذر

بر پر از فانى مکان اى طير جان
تا ببزم باقى آن گل رخان

آتش موسى پديد از سدره اش
روح صد عيسى دميد از نفخهاش

نار آن موسى ز طور آمد پديد
نار اين موسى ز جان شعله کشيد

در ميان کوه جان بس فرقها
هست ظاهر چون ثمر از ورقها

سينه اش سينا و نارش نور دوست
کف او بيضا و قلبش طور اوست

اين نه آن بيضا که ز امر آمد پديد
اين همان بيضا که امر آرد پديد

اين زمان فاران عشق آمد پديد
يار ما چون پرده از رخ بر دريد

بوى جان ميآيد اين دم بر مشام
مى ندانم کز کجا آيد مدام

اين قدر دانم که از زلفين يار
ميوزد بوئى که جان گردد نثار

نافه مشک الهى باز شد
جان ما با ياد او همراز شد

اى نسيم صبح روحانى بوز
از سباى قدس رحمانى بوز

تا ز بوى عنبرت جانهاى مست
برپرند از ارض هستى تا الست

چونکه عنقاى بقا از قاف جان
بر پريد او تا هواى لا مکان

هم بيک پر سير آفاق جهان
کرد از تاييد آن سلطان جان

باز آمد اين زمان از عرش يار
نغمه هاى او برونست از شمار

از گل رويش دى آمد چون بهار
و ز لب لعلش شب آمد چون نهار

کار عشاقان ز زلفش شد دراز
جمله معشوقان ز هجرش در نياز

گردن گُردان بمويش در کمند
صفدر يزدان ز تيرش مستمند

از لبش جانهاى عشاقان بلب
هم ز وصلش جان شاهان در طلب

از جمالش چشم جان معنوى
گشت روشن گر تو نيکو بنگرى

گر نبودى چشم او اندر جهان
چشمه هاى نور کى گشتى روان

از گلش بس گلستان آمد پديد
و ز رخش گلهاى معنى بر دميد

نار موسى نور جو در کوى او
جان عيسى روح جو از روى او

گر شبى آيد برون او از حجاب
صد جهان روشن کند چون آفتاب

ليل نبود جز ز زلف آن نگار
صبح نايد جز ز نور روى يار

شهرياران جمله اندر شهر عشق
جان نثار آوردهاند از بهر عشق

از جما ل او جمال ا لّله پديد
وز لبش دل خمر جان اندر کشيد

جملهء عالم بمويش بسته است
هم ز بهرش سينه هاشان خسته است

چون زليخاى جمال آنروى ديد
در مقام دست او دل را بريد

يک نفس از روح خود چون بر دميد
صد هزاران روح عيسى شد پديد

اين نه وصف او بود اى ذو صفات
وصف آن نورى کزو هستت حيات

گر تو بر وصف جمالش پى برى
از هزاران بحر معنى بگذرى

وصف يک پرتو که باشد اينچنين
وصف او خود چون بود ايمرد دين

چشم عاشق چون جمال او بديد
هم ز دنيا هم ز عقبى دل بريد

موج درياهاى عشق از موج او
اوج عنقاهاى عشق از اوج او

چونکه چشم تو ز چشمش نور يافت
ظلم باشد گر بغير او بتافت

چونکه نور از او گرفته چشم جان
حيف باشد گر فتد بر ديگران

چشم تو از چشم حق گشته عيان
تا نه بينى جز جمالش در جهان

سرّ اين سر بسته گفتم اى رفيق
درّ اين در خفيه سفتم اى شفيق

تا نيفتد چشم بد بر روى او
تا نيابد غير راه کوى او

همچنين در کلّ اعضا اين بدان
تا رهى از قيد اين ظلماتيان

گوش تو چون نغمهء رازش شنيد
رازهاى جانى از سازش شنيد

چونکه صنع ايزدى گشته عيان
چشم بر او کن از اين خلق جهان

گر تو با چشمش جهانرا بنگرى
بر هزاران ملک معنى پى برى

مى نبيند چشم او جز روى او
مى نپرد مرغ او جز کوى او

از وصالش جان عشاقان بسوخت
و ز فراقش نار دلها بر فروخت

پس بسوزد عاشق بيجان و سر
هم ز هجر و هم ز وصلش اى پسر

پس تو عشق حق رفيق خود بدان
تا شوى پرّان ز قيد اين جهان

عشق آن باشدکه جان فانى کنى
جان و دل در ملک باقى افکنى

سرّ اين معنى شنو گر پى برى
تا به معراج الهى بر پرى

تا که نخلت بار روحانى دهد
ميوه هاى قدس نورانى دهد

اى نسيم از زلف او عطرى بيار
اى غمام از فضل هو رشحى ببار

تا رياض جان عشاقان او
لاله هاى عشق آرد بس نکو

اين دل عاشق بود عرش اله
چونکه پاک آمد ز قيد ما سواه

چون ز حبّش بيت او معمور شد
او به بيت و بيت او مستور شد

بيت او از سنگ وگِل نبود بدان
بيت او جز دل نباشد اى جوان

چونکه قلبت پاک شد از نور او
شد مقامش چونکه آمد طور او

چونکه بيت اللّه عاشق شد‌ تمام
جلوه معشوق آمد بر دوام

باز عشق آمد حجاب عقل سوخت
خرمن عرفان و علم و فضل سوخت

چونکه غيرش نيست دربيت اى پسر
جمله حکم او بدان تو سر بسر

پس تو چشم و گوش و دست از‌ او بدان
او ببيند او بگيرد آن زمان

جان عارف مسجد اقصاى اوست
مخزن اسرار او ادناى اوست

چاره ئى اکنون ز نو بايد نمود
اين نصيحت را بجان بايد شنود

هم ز هجر و وصل هر دو در گذر
تا رسى در رفرف اصل اى پسر

تا تو در هجرى يقين در آتشى
هم ز وصلش در تب و هم ناخوشى

پاى نه بر عرصه پاک بقا
که بود غيرش در آن ميدان فنا

گر حديث کان لللّه خوانده اى
ور تو رمز ليس غيره ديده اى

پاى همت اندرين ره تو گذار
تا شوى فارغ ز وصل و هجر يار

چونکه دانستى يقين ز اسرار
جان که نباشد غير يزدان در ميان

پس ز آب جان بران خاشاک را
تا ببينى جلوه آن پاک را

تا ببينى تو وصال اندر وصال
تا ببينى در دلت نور جمال

اين بود وصلی که ضد نبود ورا
بلکه هجرش مى نباشد از ورا

وصل و هجر تو بود شرک اى پسر
گر تو دارى گوش بر پند پدر

زين دو عقبه چون هما برپر برو
تا هواى وحدت سلطان هو

ليک ترسم که بلغزد پاى تو
وهم بد پيدا شود در راى تو

واجب آمد شرح اين معنى کنم
بيخ وسواس دل از بُن بر کنم

تا نيفتى زين بيان اندر غرور
وارهى از کبر و ناز و شرّ و شور

وصل او را تو تجلّيش بدان
که شده بيچند و چون در تو عيان

نور او در تو وديعه او بود
جهد آن کن تا که او ظاهر شود

پس تو وصل او ز خود جو اى نگار
تا نه بينى بعد از اين هجران يار

مخزن کنز الهى هم توئى
ليک از غفلت پى اينان دوى

تا نگردد در تو اوصافش عيان
خويش را در هجر و گمراهى بدان

او ز جود خود نکردت بى نصيب
از صفات و اسم و رسمش اى لبيب

او زلطفش بابها بر تو گشود
تو مبند آن باب ها همچون يهود

چون شنيدى‌ ناله نى را ز عشق
اين زمان بشناس او را هم ز عشق

چون شنيدى صوت نى نائى نگر
تا نباشى بى خبر از شه مگر

چونکه نائى در جهان اغيار ديد
زان سبب نى را حجاب خود گزيد

پس تو بر در ‌ اين حجابت يکزمان
تا که جز نائى نه بينى در جهان

همچو صفدر بردران احجاب را
تا ببينى جلوه وهّاب را

همچو نى بخروش تو اندر فراق
تا که آيد نائيت اندر وثاق

چون در آيد نائى دل در خروش
سينه هاى عاشقان آيد به جوش

آتشى بفروز زين نى تو همى
تا بسوزى در جهان وصف منى

از منى چون ميم سوزد در جهان
غير نى باقى نماند در ميان

چونکه گردد چشمت از نورش بصير
غير نائى خود نبينى اى خبير

پس ز نائى بشنو اين اسرارها
تا برى بوئى از اين گلزارها

يک شرر از نار عشقش بر فروخت
خرمن هستى سلطانى بسوخت

چون جمالش پرده از رخ برکشيد
پرده اجلال سلطانان دريد

خورد چون تيرى ز مژگان نگار
بر دريد او صدر جان شهريار

تاج شاهى را ز سر آندم فکند
بنده گشت و آنگه افتاد او به بند

همچو صيدى دست صيادى فتاد
يا چه کاهى در دم بادى فتاد

گر بود پيکى رود سوى عراق
شرح گويد درد هجران و فراق

کز فراقت جان مشتاقان بسوخت
تير هجرت سينهء شاهان بدوخت

در ميان ما و تو اى شهر جان
صد هزاران قاف باشد در ميان

نيست پيکى جز که آه پر شرر
يا رود باد صبا گويد خبر

دست از نخلش بسى کوتاه ماند
جان ز هجرش بحرها از چشم راند

اى صبا از پيش جانان يکزمان
خوش بران تا کوى آن زورائيان

پس بگويش کى مدينه کردگار
چون بماندى چونکه رفت از برت يار

يار تو در حبس و زندان مبتلا
چون حسين اندر زمين کربلا

يک حسين و صد هزارانش يزيد
يک حبيب و اين همه ديو عنيد

چون کليم اندر ميان قبطيان
يا چه روح اللّه ميان سبطيان

همچو يوسف اندر افتاده بچاه
آن چهى که نبودش پايان و راه

بلبلت شد مبتلی اندر قفس
بسته شد هم زين قفس راه نفس
