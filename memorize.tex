% -*- bidi-paragraph-direction: left-to-right -*-

\documentclass[11pt]{article}

\usepackage[margin=1in,top=1.25in,bottom=1.25in]{geometry}
\usepackage{soul}
\usepackage{tabularx}
\usepackage{fontspec}
\usepackage{xunicode}
\usepackage{setspace}
\usepackage{arabxetex}

\setuldepth{sh}

\setmainfont[Ligatures=TeX]{GaramondPremrPro}[
  Path           = /Users/johnw/Library/Fonts/ ,
  Extension      = .otf ,
  BoldFont       = *-Smbd ,
  ItalicFont     = *-It ,
  BoldItalicFont = *-SmbdIt
]

\newfontfamily\arabicfont{ScheherazadeNew}[
  Path           = /Library/Fonts/Nix Fonts/vf95r7q0kpvp4nb88qnm8ljhfwsnnvbv-scheherazade-new-4.400/share/fonts/truetype/ ,
  Extension      = .ttf ,
  UprightFont    = *-Regular ,
  BoldFont       = *-Bold ,
  Script         = Arabic
]

\newenvironment{orig}
  {\begin{farsi}[voc]}
  {\end{farsi}}

\newenvironment{trans}
  {\Large\begin{spacing}{1.2}\raggedright}
  {\end{spacing}}

\newenvironment{word}
  {\begin{tabular}[t]{p{2.75in}@{\hspace{3em}}p{2.875in}}}
  {\end{tabular}}

\newcommand{\ayat}[2]{\begin{orig}#1\end{orig} & \begin{trans}#2\end{trans}}
\newcommand{\heading}[2]{\textsc{\textbf{#1}} % \ (\##2)
}
\newcommand{\define}[3]{\textfarsi[voc]{\Large
    \textbf{#1}}\hspace{3mm}{\headwordfont \large
    \textit{#2}}\hspace{3mm}{\Large #3} \\[3ex]}
\newcommand{\fulldefine}[3]{\textfarsi[voc]{\Large
    \textbf{#1}}\hspace{3mm}{\headwordfont \large
    \textit{#2}}\vspace{-1ex}\begin{quote}\Large #3\end{quote}\vspace{1ex}}

\title{
\Large
\vspace*{2in}
Selections from \\
the Sacred Writings \\
\vspace{.25in}
\fontsize{48}{36}
\begin{arab}
\end{arab}
\vspace{1in}}
\author{\LARGE Bahá’u’lláh and `Abdu'l-Bahá}
\date{}

\begin{document}

\maketitle
\thispagestyle{empty}

\newpage

\fontsize{16}{24}

\begin{word}
\ayat{
اِلهَا پَروَردِگارا مَحبوبا مَقصُودا
}{O God, O God, my Beloved, the Goal of my Desire!} \\ \ayat{
به تو آمده ام و از تو می طلبم

آنچه را كه سببِ بخششِ تو است
}{I stand before Thee and beseech Thee by reason of Thy forgiveness,} \\ \ayat{
توئی بحرِ جود و مالكِ وجود
}{O Thou Who art the Ocean of bounty and the King of existence,} \\ \ayat{
لازال

لحاظتِ علّتِ ظهورِ بخشش و عطا
}{O Thou Who hast caused both forgiveness and tenderness to appear.} \\ \ayat{
عبادِ خود را محروم منما
}{Deny not Thy servants} \\ \ayat{
و از بِساطِ قُدس و قُرب منع مفرما
}{and withhold them not from Thy holiness and nearness.} \\ \ayat{
توئی بخشنده و مهربان
}{Thou art the Forgiving and the Kind.} \\ \ayat{
لا اِلهَ اِلّا اَنتَ العَزيزُ المَنّان
}{No God is there but Thee, the Almighty, the Most Bountiful.}
\end{word}

\newpage

\begin{word}
\ayat{
ای احبّای حقّ
}{O ye the beloved of the one true God!} \\ \ayat{
از مَفازَۀ ضَیِّقِۀ نفس و هویٰ
}{Pass beyond the narrow retreats of your evil and corrupt desires,} \\ \ayat{
به فَضاهایِ مُقَدَّسِۀ اَحَدِیِّه بشتابید
}{and advance into the vast immensity of the realm of God,} \\ \ayat{
و در حَدِیقِۀ تَقدیس و تَنزیه

مَأویٰ گیرید
}{and abide ye in the meads of sanctity and of detachment,} \\ \ayat{
تا از نَفَحاتِ اَعمالِیِّه

کُلّ بَرِیِّه

به شاطیِ عِزِّ اَحَدِیِّه

تَوَجُّه نمایند
}{that the fragrance of your deeds may lead the whole of mankind to the ocean
  of God’s unfading glory.} \\ \ayat{
اَبَداً در اُمورِ دُنیا

و مَا یَتَعَلَّقُ بها
}{Forbear ye from concerning yourselves with the affairs of this world and all
  that pertaineth unto it,} \\ \ayat{
و رُؤَسایِ ظاهِرۀ آن

تَکَلُّم جائز نه
}{or from meddling with the activities of those who are its outward leaders.}
\end{word}

% \newpage

% \begin{minipage}[t]{0.48\textwidth}
% \define{صاحبان}{sá\d{h}ibán}{owners}
% \fulldefine{هدهد}{hudhud}{The hoopoe bird, often a guide in Sufi literature.}
% \end{minipage}

\newpage

\begin{word}
\ayat{
اللّه ابهی
}{} \\ \ayat{
ای متوجّه اِلی اللّه
}{O thou who art turning thy face towards God!} \\ \ayat{
چشم از جميعِ ماسِوی بَر بَند
}{Close thine eyes to all things else,} \\ \ayat{
و به مَلَکوتِ ابهی بَر گشا
}{and open them to the realm of the All-Glorious.} \\ \ayat{
آنچه خواهی از او خواه
}{Ask whatsoever thou wishest of Him alone;} \\ \ayat{
و آنچه طلبی از او طلب
}{seek whatsoever thou seekest from Him alone.} \\ \ayat{
به نظری

صد هزار حاجاتت روا نمايد
}{With a look He granteth a hundred thousand hopes,} \\ \ayat{
و به التفاتی

صد هزار درد بی درمان دوا کند
}{with a glance He healeth a hundred thousand incurable ills,} \\ \ayat{
و به انعطافی

زخم‌ها را مَرهم نهد
}{with a nod He layeth balm on every wound,} \\ \ayat{
و به نگاهی

دل‌ها را از قيدِ غم برهاند
}{with a glimpse He freeth the hearts from the shackles of grief.}
\end{word}

\newpage

\begin{word}
\ayat{
آنچه کند او کند
}{He doeth as He doeth,} \\ \ayat{
ما چه توانيم کرد
}{and what recourse have we?} \\ \ayat{
يَفعل ما يَشاء
}{He carrieth out His Will,} \\ \ayat{
و يَحکُمُ ما يُريد است
}{He ordaineth what He pleaseth.} \\ \ayat{
پس سَرِ تسليم نـِه
}{Then better for thee to bow down thy head in submission,} \\ \ayat{
و توکّل بر رَبِّ رحيم بـِه
}{and put thy trust in the All-Merciful Lord.} \\ \ayat{
والبهاء عليک

ع ع
}{}
\end{word}

\newpage

\begin{word}
\ayat{
ای جان‌فشانِ يارِ بی‌نشان
}{O heart-surrendered lover of the traceless Friend!} \\ \ayat{
هزار عارفان در جستجوی او
}{A thousand mystic knowers have wandered far in search of Him,} \\ \ayat{
ولی محروم و مهجور از روی او
}{though all remained bereft and were kept back from beholding his Face---} \\ \ayat{
امّا تو يافتی تو شناختی
}{yet thou hast found Him; thou hast recognized Him.} \\ \ayat{
تو نردِ خدمت باختی
}{You have won the contest through service} \\ \ayat{
و کار خود ساختی
}{and established yourself thereby,} \\ \ayat{
و عَلَمِ فَوز و فلاح افراختی
}{raising up the standards of fortune and well-being.} \\ \ayat{
طُرفهْ حکايتی و غَريبْ بشارتی
}{What amazing news and marvelous tidings!} \\ \ayat{
آنانکه جستند نيافتند
}{So many, who sought but found nothing,} \\ \ayat{
آنانکه نِشستند يافتند
}{whilst these others, who sought after nothing, discovered all.}
\end{word}

\newpage

\begin{word}
\ayat{
استغفر الله
}{By God!} \\ \ayat{
جستجويشان

جستجوی سيراب بود

نه تشنگان
}{Those had longed with sated hearts, devoid of hunger;} \\ \ayat{
و طلبشان

طلبِ عاقلان بود

نه عاشقان
}{theirs was a quest of knowers, not the love-stricken.} \\ \ayat{
عاقلانِ خُوشه‌چين

از سِرِّ ليلی غافلند
}{The learned who reap their harvest know nothing of Laylí's secret:} \\ \ayat{
کاين کِرامت نيست

جز مجنونِ خرمن‌سوز را
}{that the real boon is not theirs, but with Majnún and his burnt remains.} \\ \ayat{
عاشقِ نشسته بِهْ

از عاقلِ متحرّک
}{A lover in repose doth excel the deeds of those who know.} \\ \ayat{
و البهاء عليک
}{The glory of God rest upon thee.}
\end{word}

\newpage

\begin{word}
\ayat{
حکایت آورده‌اند
}{The story is told} \\ \ayat{
که عارفِ الهی با عالمِ نحوی

هَمراه شدند و هَمراز گشتند
}{of a mystic knower who went on a journey with a learned grammarian for a
  companion.} \\ \ayat{
تا رسیدند بشاطیِ بحرُ العظمة
}{They came to the shore of the Sea of Grandeur.} \\ \ayat{
عارف بی ‌تَأَمُّل تَوَسُّل فرموده

بر آب راند
}{The knower, putting his trust in God, straightway flung himself into the
  waves,} \\ \ayat{
و عالمِ نحو

چون نقشِ بر آب محو گشته

مبهوت ماند
}{but the grammarian stood bewildered and lost in thoughts that were as words
  traced upon the water.} \\ \ayat{
بانگ زد عارف

که چون عنان پیچیدی
}{The mystic called out to him, “Why dost thou not follow?”}
\end{word}

\newpage

\begin{word}
\ayat{
گفت
}{The grammarian answered,} \\ \ayat{
ای برادر چه کنم
}{“O brother, what can I do?} \\ \ayat{
چون پای رفتنم نیست

سَرْ نَهادن اولیٰ بود
}{As I dare not advance, I must needs go back again.”} \\ \ayat{
گفت
}{Then the mystic cried,} \\ \ayat{
آنچه از سیبویه و قولویه

اَخْذ نموده‌ئی

و یا از مطالبِ ابنِ حاجب

و ابنِ مالک

حَمْل فرموده ئی

بریز و از آب بِگُذَر
}{“Cast aside what thou hast learned from Síbavayh and Qawlavayh, from
  Ibn-i-Hájib and Ibn-i-Málik, and cross the water!”\vspace{15ex}} \\ \ayat{
محو میباید نه نحو این را بدان
}{With renunciation, not with grammar’s rules, one must be armed:} \\ \ayat{
گر تو محوی بی خَطَر در آب ران
}{Be nothing, then, and cross this sea unharmed.}
\end{word}

\newpage

\begin{word}
\ayat{
أَشْهَدُ يا إِلهِي
}{I bear witness, O my God,} \\ \ayat{
بِأَنَّكَ خَلَقْتَنِيْ
}{that Thou hast created me} \\ \ayat{
لِعِرْفانِكَ وَعِبادَتِكَ،
}{to know Thee and to worship Thee.} \\ \ayat{
أَشْهَدُ فِي هذا الْحِيْنِ
}{I testify, at this moment,} \\ \ayat{
بِعَجْزِيْ وَقُوَّتِكَ

وَضَعْفِيْ وَاقْتِدارِكَ
}{to my powerlessness and to Thy might,} \\ \ayat{
وَفَقْرِيْ وَغَنائِكَ
}{to my poverty and to Thy wealth.} \\ \ayat{
لا إِلهَ إِلاَّ أَنْتَ المُهَيْمِنُ القَيُّومُ}{There is none other God but Thee, the Help in Peril, the Self-Subsisting.}
\end{word}

\newpage

\begin{word}
\ayat{
هو الله

ای طالب ملکوت

به الطاف حضرت پروردگار

اميدوار باش

و از مَصائِبِ شَديدۀِ اين جهان

نااميد مگرد

الحمدلله

خدای مهربان داری

که طَبيبِ هر بيمار است

و غَمخوارِ هر مُبتَلا

پَناهِ يَتيمان است

و مُعين بيکَسان

و بينهايت مِهرِبان
}{}
\end{word}

\newpage

\begin{word}
\ayat{
اگر بدانی که قلب عبد البهاء

چه قَدر مهربان است

البتّه از شِدَّتِ فَرَح و سُرور

پَرواز نمائی

و فَريادِ واطوبىٰ

به اُوجِ آسمان رِسانی


والبهاو عَلَیکْ

عبد البهاء عبّاس
}{}
\end{word}

\newpage

% }{} \\ \ayat{
\begin{word}
\ayat{
زبانِ خِرَد میگوید
}{The Tongue of Wisdom proclaimeth:} \\ \ayat{
هر که دارایِ من نباشد

دارایِ هیچ نه
}{He that hath Me not is bereft of all things.} \\ \ayat{
از هر چه هست بگذرید

و مرا بیابید
}{Turn ye away from all that is on earth and seek none else but Me.} \\ \ayat{
منم آفتابِ بینش

و دریایِ دانش
}{I am the Sun of Wisdom and the Ocean of Knowledge.} \\ \ayat{
پژمردگان را تازه نمایم

و مردگان را زنده کنم
}{I cheer the faint and revive the dead.} \\ \ayat{
منم آن روشنائی

که راهِ دیده بنمایم
}{I am the guiding Light that illumineth the way.} \\ \ayat{
و منم شاهبازِ دستِ بینیاز
}{I am the royal Falcon on the arm of the Almighty.} \\ \ayat{
پرِ بستگان را بگشایم

و پرواز بیاموزم
}{I unfold the drooping wings of every broken bird and start it on its
  flight.}
\end{word}

\newpage

% }{} \\ \ayat{
\begin{word}
\ayat{
و بِه قَدَمِ یَقین
}{Thus with steadfast steps} \\ \ayat{
در صِراطِ حَقُّ الیقین

قَدَم گذاریم
}{we may tread the Path of certitude,} \\ \ayat{
که لِعَلّْ نسیمِ رضا
}{that perchance the breeze that bloweth
from the meads of the good-pleasure of God} \\ \ayat{
از ریاضِ قبولِ الهی بِوَزَد
}{may waft upon us the sweet savours
    of divine acceptance,} \\ \ayat{
و این فانیان را
}{and cause us, vanishing mortals that we are,} \\ \ayat{
بملکوتِ جاودانی رِساند
}{to attain unto the Kingdom of everlasting glory.}
\end{word}

\end{document}
