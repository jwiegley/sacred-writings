% -*- bidi-paragraph-direction: left-to-right -*-

\documentclass[11pt]{article}

\usepackage[margin=0.5in,top=1.25in,bottom=0in]{geometry}
\usepackage{soul}
\usepackage{tabularx}
\usepackage{fontspec}
\usepackage{xunicode}
\usepackage{setspace}
\usepackage{arabxetex}
\usepackage{geometry}
\usepackage{morefloats}

\setuldepth{sh}
\geometry{paperwidth=120mm, paperheight=2380pt, left=0pt, top=40pt,
  textwidth=280pt, marginparsep=0pt, marginparwidth=0pt,
  textheight=2380pt, footskip=0pt}
\setcounter{totalnumber}{100}

\setmainfont[Ligatures=TeX]{GaramondPremrPro}[
  Path           = /Users/johnw/Library/Fonts/ ,
  Extension      = .otf ,
  BoldFont       = *-Smbd ,
  ItalicFont     = *-It ,
  BoldItalicFont = *-SmbdIt
]

\newfontfamily\arabicfont{Scheherazade}[
  Path        = /Users/johnw/Library/Fonts/ ,
  Extension   = .ttf ,
  UprightFont = *RegOT ,
  Script      = Arabic
]

\newfontfamily\headwordfont[Ligatures=TeX]{Georgia}[
  Path           = /Users/johnw/Library/Fonts/ ,
  Extension      = .ttf ,
  UprightFont    = * ,
  BoldFont       = * Bold ,
  ItalicFont     = * Italic ,
  BoldItalicFont = * Bold Italic ,
  Script         = Arabic
]

\newenvironment{orig}
  {\begin{farsi}[voc]}
  {\end{farsi}}

\newenvironment{trans}
  {\Large\begin{spacing}{1.2}\raggedright}
  {\end{spacing}}

\newenvironment{word}{}{}

\newcommand{\ayat}[1]{\vspace{4ex}\begin{orig}#1\end{orig}}
\newcommand{\heading}[2]{\textsc{\textbf{#1}} % \ (\##2)
}
\newcommand{\define}[3]{\textfarsi[voc]{\Huge
    \textbf{#1}}\hspace{3mm}{\headwordfont \large
    \textit{#2}}\hspace{3mm}{\Large #3} \\[3ex]}
\newcommand{\fulldefine}[3]{\textfarsi[voc]{\Huge
    \textbf{#1}}\hspace{3mm}{\headwordfont \large
    \textit{#2}}\vspace{-1ex}\begin{quote}\Large #3\end{quote}\vspace{1ex}}

\begin{document}

\fontsize{24}{32}

\thispagestyle{empty}

\begin{word}
\ayat{
هو الله
}

\ayat{
رَشحِ عَما از جَذبِۀ ما میریزد

سِرِّ وَفا از نَغمِۀ ما میریزد
}

\ayat{
از بادِ صَبا مُشکِ خَطا گَشته پدید

وین نَفحِۀ خوش از جَعدِۀ ما میریزد
}

\ayat{
شَمسِ طَراز از طَلعَتِ حَقّ کرده طُلوع

سرّ حقیقت بین کَز وَجهِۀ ثا میریزد
}

\ayat{
بَحرِ صَفا از موجِ لِقا کرده خُروش

وین طُرفِهْ عطا از جذبۀ ما میریزد
}

\ayat{
گَنجینۀ حُبّ در سینۀ فا گشته نَهان

زین گَنجِ مُحَبَّت دُرِّ وفا ميريزد
}

\ayat{
بِهجَتِ مُل از نَظرِۀ گُل شد ظاهر

این لَحنِ مَلیح از نغمۀ را میریزد
}

\ayat{
نُقرِۀ ناقوریْ جذبۀ لاهوتی

این هر دو بیک نفحه از جَوِّ سَما میریزد
}

\ayat{
دَورِ اَنَا هُو از چِهرِۀ ما کرده بُروز

کَورِ هُوَ هُو از طفحۀ ما میریزد
}

\ayat{
کوثرِ حَقّ از حقّۀ دِل گشته هُویدا

این ساغرِ شَهْد از لَعلِ بها میریزد
}

\ayat{
یومِ خدا از جِلوِۀ رَبّ شد کامل

این نَغزِ حَدیث از غَنِّۀ طا میریزد
}

\ayat{
طَفحِ بهائی بین

رشح عمائی بین

کِاین جُمله زِ یک نغمه

از لَحنِ خدا میریزد
}

\ayat{
ماهیِ سَرمَد بین

طَلعِ مُنَزَّه بین

صَدرِ مُمَرَّد بین

کز عَرشِ عَلا میریزد
}

\ayat{
نَخلِۀ طوبیٰ بین

رنّۀ ورقا بین

غنّۀ ابهی بین

کَز لَمْعِ صفا میریزد
}

\ayat{
آهنگِ عَراقی بین

دَفِّ نَوائی بین

کَفِّ الهی بین

کز ضَربۀ ما میریزد
}

\ayat{
طلعةِ لاهوتی بین

حوریِ هاهوتی بین

جلوۀ ناسوتی بین

کز ساحت ما میریزد
}

\ayat{
وجهۀ باقی بین

چهرۀ ساقی بین

رَقِّ زُجاجی بین

کز کوبۀ با میریزد
}

\ayat{
آتشِ موسیٰ بین

بَیضِۀ بَیضا بین

سینۀ سینا بین

کز کفّۀ ما میریزد
}

\ayat{
نالِۀ مَستان بین

سبزۀ بُستان بین

جذبۀ هَستان بین

کز صَحنِ لقا میریزد
}

\ayat{
وجهۀ هائی بین

طُرزۀ بائی بین

نظرۀ هائی بین

کز کِلکِ بها میریزد
}

\ayat{
طفحِ طَهور است این

رَشحِ طَهور است این

غَنِّ طُیور است این

کز عِینِ فَنا میریزد
}
\end{word}

\end{document}
