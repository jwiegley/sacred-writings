% -*- bidi-paragraph-direction: left-to-right -*-

% -*- bidi-paragraph-direction: left-to-right -*-

\documentclass[11pt]{article}

\usepackage[margin=1in,top=1.25in,bottom=1.25in]{geometry}
\usepackage{soul}
\usepackage{tabularx}
\usepackage{longtable}
\usepackage{fontspec}
\usepackage{xunicode}
\usepackage{setspace}
\usepackage{arabxetex}

\setuldepth{sh}

\setmainfont[Ligatures=TeX]{GaramondPremrPro}[
  Path           = /Users/johnw/Library/Fonts/ ,
  Extension      = .otf ,
  BoldFont       = *-Smbd ,
  ItalicFont     = *-It ,
  BoldItalicFont = *-SmbdIt
]

\newfontfamily\arabicfont{Scheherazade}[
  Path        = /Users/johnw/Library/Fonts/ ,
  Extension   = .ttf ,
  UprightFont = *RegOT ,
  Script      = Arabic
]

\newfontfamily\headwordfont[Ligatures=TeX]{Georgia}[
  Path           = /Users/johnw/Library/Fonts/ ,
  Extension      = .ttf ,
  UprightFont    = * ,
  BoldFont       = * Bold ,
  ItalicFont     = * Italic ,
  BoldItalicFont = * Bold Italic ,
  Script         = Arabic
]

\newenvironment{orig}
  {\begin{farsi}[voc]}
  {\end{farsi}}

\newenvironment{trans}
  {\Large\begin{spacing}{1.2}\raggedright}
  {\end{spacing}}

\newenvironment{word}
  {\begin{longtable}[t]{p{2.75in}@{\hspace{3em}}p{2.875in}}}
  {\end{longtable}}

\newcommand{\ayat}[2]{\begin{orig}#1\end{orig} & \begin{trans}#2\end{trans}}
\newcommand{\heading}[2]{\textsc{\textbf{#1}} % \ (\##2)
}
\newcommand{\define}[3]{\textfarsi[voc]{\Huge
    \textbf{#1}}\hspace{3mm}{\headwordfont \large
    \textit{#2}}\hspace{3mm}{\Large #3} \\[3ex]}
\newcommand{\fulldefine}[3]{\textfarsi[voc]{\Huge
    \textbf{#1}}\hspace{3mm}{\headwordfont \large
    \textit{#2}}\vspace{-1ex}\begin{quote}\Large #3\end{quote}\vspace{1ex}}

\begin{document}

\fontsize{24}{32}

\begin{word}
\ayat{به بلندای خیال}{Beyond the Limits of Imagination

by Said Reza'i} \vspace{-1ex}\\

\ayat{در قَفَس می مانم

تا تو پَرواز کنی

تا که پِژواکِ صدایم باشی

در جهانی که صدائی نشنید}{
\vspace{1em}
I remain caged

so you may take your flight

and become my voice’s echo

in a world that hears no voice.} \vspace{-1ex}\\

\ayat{من اگر در بندم

از صدایِ زنجیر

تو بساز آوازی}{
\vspace{1em}
Though I be chained up,

use the sound of my chains

to compose your melody:
} \vspace{-1ex}\\

\ayat{واژه هائی از عشق

نغمۀ آزادی

یک ترانه از صُلح

بهرِ هر سرخ و سیاه

یا که هر زرد و سِپید}{
\vspace{1em}
Words of love;

A paean of freedom;

A song of peace:

for every black and red one,

for every white and yellow.
} \vspace{-1ex}\\

\ayat{طَرْحی از پَیوَستَن

دست هائی در هم

نه که مُشتی بر هم}{
\vspace{1em}
A plan of connection:

Hand in hand;

not fist on fist.
} \vspace{-1ex}\\

\ayat{پِلِه هائی از امید

به بلندای خیال

راهی از قَطعِۀ خاک

در گُذَر از افلاک

رو به آن سِرِّ وجود}{
\vspace{1em}
A stairway of hope

beyond the limits of imagination.

A path from this plot of earth

to beyond the heavens;

towards the Mystery of existence.
} \vspace{-1ex}\\

\ayat{}{*} \vspace{-1ex}\\

\ayat{گر به پُشتِ میله

مانده ام در پیله

تو بشو پروانه}{
\vspace{1em}
If, behind bars I am cocooned,

may you become the butterfly.
} \vspace{-1ex}\\

\ayat{در گُذَر از هر باغ

رنگ و عَطْرَش دریاب}{
\vspace{1em}
Wing your way through every garden!

Revel in the colors and bouquets!
} \vspace{-1ex}\\

\ayat{عَطْرها در کِثرَت

همه گلها خوش بو}{
\vspace{1em}
For the perfumes are manifold;

and every flower lends its sweetness...
} \vspace{-1ex}\\

\ayat{نقش ها رنگارنگ

همه رنگِ وحدت

رازِ کَثرَت دریاب

راهِ وحدت بِسپار}{
\vspace{1em}
A pattern of colors,

each some shade of unity;

Learn the secret of diversity!

Pave the path to oneness.
} \vspace{-1ex}\\

\ayat{رازهای این باغ

در دلِ خاک ببین}{
\vspace{1em}
The mysteries of this garden

are found in the soil’s heart:
} \vspace{-1ex}\\

\ayat{یک نِگَه بر خورشید

باد و آب و مهتاب

باغبانی پُرکار}{
\vspace{1em}
One eye upon the sun;

wind and rain and light of moon;

the gardener’s travail.
} \vspace{-1ex}\\

\ayat{در یکی گوشۀ باغ

کِرمِ شب تابی بین}{
\vspace{1em}
In a corner of this garden

observe the firefly:
} \vspace{-1ex}\\

\ayat{در حِصارِ ناگُزیر

در گُریز از ظُلمت

نورِ خود می تابد

سَهْمِ خود می جوید}{
\vspace{1em}
Always upon the fence

he escapes the darkness

by his glimmering alone

and seeks his share.
} \vspace{-1ex}\\

\ayat{*}{} \vspace{-1ex}\\

\ayat{سَهْمِ تو از دُنیا

یک صدا و آواز

کوششی در پَرواز

دست هائی پُربار

فکر و اندیشۀ ناب

پایمَردی در راه}{
\vspace{1em}
Your share of the world:

A song and a melody;

the effort of flight;

hands ever-occupied;

pure thoughts and reflections;

steadfastness along the way.
} \vspace{-1ex}\\

\ayat{*}{} \vspace{-1ex}\\

\ayat{و بدان ، در میانِ دیوار

یا که زنجیر و حصار

من تو را می شِنَوَم

دستِ تو می بینم

ذِهنِ تو می یابم}{
\vspace{1em}
Know that whether from walls

or chains or fences:

I hear you.

I see your hands.

I discern your thoughts.
} \vspace{-1ex}\\

\ayat{و بدانم

که تَرَک بردارد

هر حصار و پیله

صدهزاران میله}{
\vspace{1em}
And I know

that cracks will form

in every wall and cocoon

and hundred thousand bars...
} \vspace{-1ex}\\

\ayat{و در آغوش کِشیم

روزِ دیدار و وصال

هر چه خوش بختی را

طَعْمِ آزادی را}{
\vspace{1em}
And at last we shall embrace:

the day of our reunion;

every good and fortunate thing;

the taste of freedom.
}

\end{word}

\end{document}
